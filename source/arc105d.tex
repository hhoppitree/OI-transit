\documentclass[UTF8,12pt,a4paper]{ctexart} % 目录没改
\usepackage[utf8]{inputenc}
\usepackage{color}
\usepackage{array}
\usepackage{diagbox}
\usepackage{multicol}
\usepackage{multirow}
\usepackage{CJKfntef}
\usepackage{booktabs}
\usepackage{fancyhdr}
\usepackage{graphicx}
\usepackage{lastpage}
\usepackage{indentfirst}
\usepackage{amsmath,amssymb}
\usepackage{tabu}
\usepackage[colorlinks,linkcolor=blue,anchorcolor=blue,citecolor=green,bookmarks=true]{hyperref}
% \usepackage[breaklinks,colorlinks,linkcolor=black,citecolor=black,urlcolor=black]{hyperref}
\usepackage{bookmark}
\usepackage{inconsolata}
\usepackage{geometry}
\usepackage{titlesec}
\usepackage{paralist}
\usepackage{multirow}
\usepackage{booktabs} % for much better looking tables
\usepackage{array} % for better arrays (eg matrices) in maths
\usepackage{paralist} % very flexible & customisable lists (eg. enumerate/itemize, etc.)
\usepackage{verbatim} % adds environment for commenting out blocks of text & for better verbatim
\usepackage{subfig} % make it possible to include more than one captioned figure/table in a single float
\usepackage{xcolor}
\usepackage{endnotes}
\usepackage{multirow}
\usepackage{notoccite}
\usepackage{longtable}
\usepackage{geometry}
\usepackage{multicol}
\usepackage{multirow}
\usepackage{tabu}
\usepackage{xeCJK}
\usepackage{CJK}                   
\usepackage{CJKfntef}        
\usepackage{fancyhdr}               
\usepackage{graphicx}                 
\usepackage{lastpage}    
\usepackage{listings}
\usepackage{fancybox}
\usepackage{xcolor}
\usepackage{fontspec}
\usepackage{layout}
\usepackage{titletoc}
\usepackage{listings}
\usepackage{color}
\usepackage{xcolor}
\usepackage{ctex}
\usepackage{mathrsfs}
\usepackage{hyperref}
\usepackage{xeCJKfntef, multicol}
\setlength{\parindent}{2em}
\geometry{left=2.45cm,right=2.45cm,top=2.75cm,bottom=2.6cm}
%\setcounter{secnumdepth}{0}
\let\itemize\compactitem
\let\enditemize\endcompactitem
\let\enumerate\compactenum
\let\endenumerate\endcompactenum
\let\description\compactdesc
\let\enddescription\endcompactdesc

\lstset{
	basicstyle={      
		\color{black}
		\fontspec{Consolas}
	},
	keywordstyle={
		\color{blue}
		\fontspec{Consolas}
	},
	numberstyle={
		\color{black}
		\textbf
	},
	rulecolor=\color{blue},
	numbers=left,                               
	frame=single,                            
	frameround=tttt,
	morekeywords={Sample, Input, Output},   % 可以手动添加关键字
}
\setmonofont{Consolas}
\newcommand{\stress}[1]{\textbf{\CJKunderdot{#1}}}
\newfontfamily\code{Consolas}

\definecolor{mys}{rgb}{1,0.2,0}%一个颜色
\definecolor{codegray}{rgb}{0.5,0.5,0.5}
\lstset{
	numberstyle=\ttfamily\color{codegray}
}

\linespread{1.5}%修改行距

\begin{document}
	\fontsize{12pt}{12pt}\selectfont
	%自行调整字体大小,适应排版
	%需要调整到合适的大小才能好看
	
	\newpage
	\pagestyle{fancy}
	\lhead{\footnotesize \songti{Let's Play Nim}}
	\cfoot{\footnotesize 第 \thepage \ 页\qquad  共  \pageref{LastPage} 页}
	
	\phantomsection
	\addcontentsline{toc}{section}{【ARC105D】Let's Play Nim}
	\section*{【ARC105D】Let's Play Nim}
	\rhead{\footnotesize \songti{ARC105D}}
	
	\phantomsection
	\addcontentsline{toc}{subsection}{题目描述}
	\subsection*{【题目描述】}
	
	有 $n$ 堆石头和 $n$ 个盘子,其中第 $i$ 堆石头中有 $a_i$ 个石子,先后手轮流操作,此时若还有剩余的石子堆则任取一堆将其\stress{全部}放入一个盘子中,否则从任意一个盘子中\stress{任意}取走正整数个石子,无法操作者输,求先后手谁有必胜策略。
	
	$n\le10^5$。
	
	\phantomsection
	\addcontentsline{toc}{subsection}{思路}
	\subsection*{【思路】}
	
	Step \#1:将游戏分为两个阶段,有剩余石子堆的和没有剩余石子堆的情况,则后半部分为一个 $\text{Nim}$ 游戏,考虑其先手必胜的充分必要条件。
	
	Step \#2:当 $n$ 为奇数时,后手成为第二个阶段的先手;当 $n$ 为偶数时,先手成为第二个阶段的先手,所以双方对自己在第一阶段最终的目标很明确。
	
	Step \#3:对 $n$ 分奇偶讨论,发现 $n$ 为奇数时后手必胜,考虑能否将其推广至 $n$ 为偶数的情况。
	
	\phantomsection
	\addcontentsline{toc}{subsection}{题解}
	\subsection*{【题解】}
	
	学过基础博弈论(Nim 游戏)的都知道,该题可以分为两个子问题,$n$ 为偶数和 $n$ 为奇数的情况:
	
	当 $n$ 为奇数时,相当于先手可以将一个盘子中的石子数加上一堆石头中石子的数量,目标是使得最终所有盘子中的石子数的异或和为 $0$。
	
	而后手可以每次选择石子数最大的一堆石头和盘子,将这一堆石头放入这个盘子中,则最终一定有一个盘子中石子的数量大于其余所有盘子中的石子数量总和,异或和必然不为 $0$,所以后手必胜。
	
	当 $n$ 为偶数时,先手的目标是使得最终所有盘子中的石子数的异或和\stress{不}为 $0$,如果对于每种石子数量,均有偶数堆石头满足其中有这么多个石子,那么显然后手必胜(先手取什么后手也取什么),否则先手必胜(策略同 $n$ 为奇数时后手的策略)。
	
	时间复杂度 $\mathcal{O}(n)$。
	
\end{document}
\documentclass[UTF8,12pt,a4paper]{ctexart} % 目录没改
\usepackage[utf8]{inputenc}
\usepackage{color}
\usepackage{array}
\usepackage{diagbox}
\usepackage{multicol}
\usepackage{multirow}
\usepackage{CJKfntef}
\usepackage{booktabs}
\usepackage{fancyhdr}
\usepackage{graphicx}
\usepackage{lastpage}
\usepackage{indentfirst}
\usepackage{amsmath,amssymb}
\usepackage{tabu}
\usepackage[colorlinks,linkcolor=blue,anchorcolor=blue,citecolor=green,bookmarks=true]{hyperref}
% \usepackage[breaklinks,colorlinks,linkcolor=black,citecolor=black,urlcolor=black]{hyperref}
\usepackage{bookmark}
\usepackage{inconsolata}
\usepackage{geometry}
\usepackage{titlesec}
\usepackage{paralist}
\usepackage{multirow}
\usepackage{booktabs} % for much better looking tables
\usepackage{array} % for better arrays (eg matrices) in maths
\usepackage{paralist} % very flexible & customisable lists (eg. enumerate/itemize, etc.)
\usepackage{verbatim} % adds environment for commenting out blocks of text & for better verbatim
\usepackage{subfig} % make it possible to include more than one captioned figure/table in a single float
\usepackage{xcolor}
\usepackage{endnotes}
\usepackage{multirow}
\usepackage{notoccite}
\usepackage{longtable}
\usepackage{geometry}
\usepackage{multicol}
\usepackage{multirow}
\usepackage{tabu}
\usepackage{xeCJK}
\usepackage{CJK}                   
\usepackage{CJKfntef}        
\usepackage{fancyhdr}               
\usepackage{graphicx}                 
\usepackage{lastpage}    
\usepackage{listings}
\usepackage{fancybox}
\usepackage{xcolor}
\usepackage{fontspec}
\usepackage{layout}
\usepackage{titletoc}
\usepackage{listings}
\usepackage{color}
\usepackage{xcolor}
\usepackage{ctex}
\usepackage{mathrsfs}
\usepackage{hyperref}
\usepackage{xeCJKfntef, multicol}
\setlength{\parindent}{2em}
\geometry{left=2.45cm,right=2.45cm,top=2.75cm,bottom=2.6cm}
%\setcounter{secnumdepth}{0}
\let\itemize\compactitem
\let\enditemize\endcompactitem
\let\enumerate\compactenum
\let\endenumerate\endcompactenum
\let\description\compactdesc
\let\enddescription\endcompactdesc

\lstset{
	basicstyle={      
		\color{black}
		\fontspec{Consolas}
	},
	keywordstyle={
		\color{blue}
		\fontspec{Consolas}
	},
	numberstyle={
		\color{black}
		\textbf
	},
	rulecolor=\color{blue},
	numbers=left,                               
	frame=single,                            
	frameround=tttt,
	morekeywords={Sample, Input, Output},   % 可以手动添加关键字
}
\setmonofont{Consolas}
\newcommand{\stress}[1]{\textbf{\CJKunderdot{#1}}}
\newfontfamily\code{Consolas}

\definecolor{mys}{rgb}{1,0.2,0}%一个颜色
\definecolor{codegray}{rgb}{0.5,0.5,0.5}
\lstset{
	numberstyle=\ttfamily\color{codegray}
}

\linespread{1.5}%修改行距

\begin{document}
	\fontsize{12pt}{12pt}\selectfont
	%自行调整字体大小,适应排版
	%需要调整到合适的大小才能好看
	
	\newpage
	\pagestyle{fancy}
	\lhead{\footnotesize \songti{Keep Graph Disconnected}}
	\cfoot{\footnotesize 第 \thepage \ 页\qquad  共  \pageref{LastPage} 页}
	
	\phantomsection
	\addcontentsline{toc}{section}{【ARC105E】Keep Graph Disconnected}
	\section*{【ARC105E】Keep Graph Disconnected}
	\rhead{\footnotesize \songti{ARC105E}}
	
	\phantomsection
	\addcontentsline{toc}{subsection}{题目描述}
	\subsection*{【题目描述】}
	
	给定一张 $n$ 个点,$m$ 条边的简单无向图,保证初始时点 $1$ 和点 $n$ 不连通,先后手轮流加边,在过程中要保证点 $1$ 和点 $n$ 不能连通且图为简单图,不能操作者负,求先后手谁有必胜策略。
	
	\phantomsection
	\addcontentsline{toc}{subsection}{思路}
	\subsection*{【思路】}
	
	Step \#1:假设最后点 $1$ 所在的连通块大小为 $x$,则最后点 $n$ 所在连通块大小为 $n-x$。
	
	Step \#2:发现此时输赢取决于 $\dfrac{n(n-1)}{2}-x(n-x)-m$ 的奇偶性。
	
	Step \#3:当 $n$ 为奇数时,此时此式的奇偶性确定,所以只需考虑 $n$ 为偶数时的情形。
	
	Step \#4:考察初始时点 $1$ 和点 $n$ 所在连通块的大小与答案之间的关系。
	
	\phantomsection
	\addcontentsline{toc}{subsection}{题解}
	\subsection*{【题解】}
	
	发现最后的图一定是两个完全图拼在一起构成的,所以假设最后点 $1$ 所在的连通块大小为 $x$,则输赢取决于 $\dfrac{n(n-1)}{2}-x(n-x)-m$ 的奇偶性。
	
	当 $n$ 为奇数时此式的奇偶性已经确定,考虑 $n$ 为偶数时的情况,此时输赢取决于 $x$ 的奇偶性。
	
	假设初始时点 $1$ 和点 $n$ 所在连通块的大小的奇偶性不同,那么此时除了这两个连通块以外肯定有奇数个奇连通块,先手先将一个奇连通块分配给其中一个连通块,然后每一步都“抵消”后手的操作即可。
	
	当初始时点 $1$ 和点 $n$ 所在连通块的大小的奇偶性相同时,先手或后手总是有一方可以保持当前的连通块大小的奇偶性不变,而双方肯定有一方会因此胜利,从而采取这种策略,所以\stress{最终}点 $1$ 所在连通块的大小的奇偶性和以\stress{初始时}点 $1$ 所在连通块的大小的奇偶性相同。
	
	dfs 一遍即可,时间复杂度 $\mathcal{O}(n+m)$
	
\end{document}
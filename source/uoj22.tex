\documentclass[UTF8,12pt,a4paper]{ctexart} % 目录没改
\usepackage[utf8]{inputenc}
\usepackage{color}
\usepackage{array}
\usepackage{diagbox}
\usepackage{multicol}
\usepackage{multirow}
\usepackage{CJKfntef}
\usepackage{booktabs}
\usepackage{fancyhdr}
\usepackage{graphicx}
\usepackage{lastpage}
\usepackage{indentfirst}
\usepackage{amsmath,amssymb}
\usepackage{tabu}
\usepackage[colorlinks,linkcolor=blue,anchorcolor=blue,citecolor=green,bookmarks=true]{hyperref}
% \usepackage[breaklinks,colorlinks,linkcolor=black,citecolor=black,urlcolor=black]{hyperref}
\usepackage{bookmark}
\usepackage{inconsolata}
\usepackage{geometry}
\usepackage{titlesec}
\usepackage{paralist}
\usepackage{multirow}
\usepackage{booktabs} % for much better looking tables
\usepackage{array} % for better arrays (eg matrices) in maths
\usepackage{paralist} % very flexible & customisable lists (eg. enumerate/itemize, etc.)
\usepackage{verbatim} % adds environment for commenting out blocks of text & for better verbatim
\usepackage{subfig} % make it possible to include more than one captioned figure/table in a single float
\usepackage{xcolor}
\usepackage{endnotes}
\usepackage{multirow}
\usepackage{notoccite}
\usepackage{longtable}
\usepackage{geometry}
\usepackage{multicol}
\usepackage{multirow}
\usepackage{tabu}
\usepackage{xeCJK}
\usepackage{CJK}                   
\usepackage{CJKfntef}        
\usepackage{fancyhdr}               
\usepackage{graphicx}                 
\usepackage{lastpage}    
\usepackage{listings}
\usepackage{fancybox}
\usepackage{xcolor}
\usepackage{fontspec}
\usepackage{layout}
\usepackage{titletoc}
\usepackage{listings}
\usepackage{color}
\usepackage{xcolor}
\usepackage{ctex}
\usepackage{mathrsfs}
\usepackage{hyperref}
\usepackage{xeCJKfntef, multicol}
\setlength{\parindent}{2em}
\geometry{left=2.45cm,right=2.45cm,top=2.75cm,bottom=2.6cm}
%\setcounter{secnumdepth}{0}
\let\itemize\compactitem
\let\enditemize\endcompactitem
\let\enumerate\compactenum
\let\endenumerate\endcompactenum
\let\description\compactdesc
\let\enddescription\endcompactdesc

\lstset{
	basicstyle={      
		\color{black}
		\fontspec{Consolas}
	},
	keywordstyle={
		\color{blue}
		\fontspec{Consolas}
	},
	numberstyle={
		\color{black}
		\textbf
	},
	rulecolor=\color{blue},
	numbers=left,                               
	frame=single,                            
	frameround=tttt,
	morekeywords={Sample, Input, Output},   % 可以手动添加关键字
}
\setmonofont{Consolas}
\newcommand{\stress}[1]{\textbf{\CJKunderdot{#1}}}
\newfontfamily\code{Consolas}

\definecolor{mys}{rgb}{1,0.2,0}%一个颜色
\definecolor{codegray}{rgb}{0.5,0.5,0.5}
\lstset{
	numberstyle=\ttfamily\color{codegray}
}

\linespread{1.5}%修改行距

\begin{document}
	\fontsize{12pt}{12pt}\selectfont
	%自行调整字体大小,适应排版
	%需要调整到合适的大小才能好看
	
	\newpage
	\pagestyle{fancy}
	\lhead{\footnotesize \songti{外星人}}
	\cfoot{\footnotesize 第 \thepage \ 页\qquad  共  \pageref{LastPage} 页}
	
	\phantomsection
	\addcontentsline{toc}{section}{【UR \#1】外星人}
	\section*{【UR \#1】外星人}
	\rhead{\footnotesize \songti{UR \#1}}
	
	\phantomsection
	\addcontentsline{toc}{subsection}{题目描述}
	\subsection*{【题目描述】}
	
	给定正整数 $x$ 与数列 $a_n$,求将 $a$ 重新排列后 $x\bmod a_1\bmod a_2\bmod\cdots\bmod a_n$ 的最大值,并输出方案数,对 $998244353$ 取模。
	
	$n\le1000$,$x,a_i\le5000$。
	
	\phantomsection
	\addcontentsline{toc}{subsection}{思路}
	\subsection*{【思路】}
	
	Step \#1:如果 $x$ 已经小于 $a_i$ 了,那么 $a_i$ 就是没用的。
	
	Step \#2:第一问可以设 $f_i$ 表示 $x=i$ 时最终的最大结果,简单 $\text{DP}$ 得出。
	
	Step \#3:第二问的问题在于不知道当前剩余的数所构成的集合。
	
	Step \#4:考虑当最新一次的有效的 $\bmod\;x$ 确定后,假设当前 $a\rightarrow b$,则可以先将 $a\sim b$ 之间的数插入后方。
	
	\phantomsection
	\addcontentsline{toc}{subsection}{题解}
	\subsection*{【题解】}
	
	当 $x<a_i$ 时,就可以把 $a_i$ 扔进垃圾桶,所以设 $f_i$ 为 $x=i$ 时最终的最大结果,就可以对于每一个状态枚举下一个 $<i$ 的 $a_j$ 进行转移即可。
	
	设 $g_i$ 为此时的方案数,转移的时候 $g_{i\bmod a_j}$ 时只需从 $g_i$ 乘上一个组合数的贡献系数(将新产生的 $j$ 与 $i$ 之间的垃圾在剩余的位置里插入)即可,时间复杂度 $\mathcal{O}(nx)$。
	
	
\end{document}